\NeedsTeXFormat{LaTeX2e}
\documentclass[12pt,a4paper,twoside]{report}
% zur Kontrolle des Umbruchs Klassenoption draft verwenden
%
%
%%% Ablauf in der Konsole:
%
% pdflatex main
% pdflatex main
% biber main
% xindy -L german-din -C utf8 -I xindy -M main -t main.glg -o main.gls main.glo
% texindy main.idx
% pdflatex main
% pdflatex main

\usepackage[ngerman]{babel}
\usepackage[utf8]{inputenc}
%\usepackage[T1]{fontenc}
\usepackage{graphicx}
\usepackage{subfig}
\usepackage{float}
\usepackage{latexsym}
\usepackage{amsmath,amssymb}
%\usepackage{ifpdf}
%\usepackage{array}
\usepackage{listings}  % Listings, Codebeschreibungen
\usepackage{blindtext}

\usepackage{dcolumn} % an der Dezimale ausgerichtete Spalten in Tabellen
\newcolumntype{.}{D{.}{.}{-1}}
\newcolumntype{d}[1]{D{.}{.}{#1}}
\newcolumntype{,}{D{,}{\mathord\mathcomma}{-1}} % Siehe Beschreibung

% Platzierung von auch großen Grafiken
\setcounter{topnumber}{50}
\setcounter{bottomnumber}{50}
\setcounter{totalnumber}{100}
\renewcommand{\topfraction}{.99}
\renewcommand{\bottomfraction}{.99}
\renewcommand{\textfraction}{.00}
\renewcommand{\floatpagefraction}{.99}

\usepackage{hyperref}
\usepackage[xindy]{glossaries}
\GlsSetXdyCodePage{duden-utf8}
\loadglsentries{main.glossary}
\makeglossaries
\glstoctrue
% Auszuführender Befehl: (nicht makeglossaries verwenden!)
% xindy -L german-din -C utf8 -I xindy -M main -t main.glg -o main.gls main.glo

%% Bibtex-optionen definieren, unterscheiden sich ggf. in Sortieroptionen
% no sorting, in order of appearance
\usepackage[style=numeric,backend=biber,sorting=none,sortlocale=de]{biblatex}
\addbibresource{main.bib}

% Definition der Seite
\parindent0em
\evensidemargin0cm
\oddsidemargin0cm
\textheight22cm
\textwidth16cm

\usepackage[xindy]{imakeidx}
%\usepackage{makeidx} % Paket zur Stichwortverzeichniserstellung
\makeindex % Alle Worte/Zeichen/Texte die mit \index{...} markiert werden

\begin{document}
% Keine Seitenzahlen im Vorspann
\pagestyle{empty}

% Titelblatt der Arbeit
\begin{titlepage}
  \begin{center}\large
    \vspace*{1cm}
    %% Bitte anpassen
    Studienarbeit oder Bachelorarbeit (anpassen)\\
    des \\
    Studienganges Elektrotechnik
    \\
    an der Dualen Hochschule Baden-Württemberg Karlsruhe
    \vspace*{2cm}

    {\huge Wichtige Studienarbeit}\\
    \vspace*{2cm}
    von\\
    \vspace*{.5cm}
    Mein Name\\
    \vspace*{.5cm}
    \today\ (Besser Abgabedatum)\\
    \vspace*{2cm}
    %\includegraphics[height=8cm]{vorlage_diplomarbeit_bild.eps}
    \vspace*{2cm}
  \end{center}
  \vspace*{2cm}

  \begin{tabbing}
    Betreuer der Dualen Hochschule \= \hspace{5mm} \= Prof. Dr.-Ing. Vorname langerlangernachname \kill
    Bearbeitungszeitraum \> \> von bis Dauer\\
    Matrikelnummer, Kurs \> \> Bitte eintragen \\
    Ausbildungsfirma \> \> Wenn relevant, sonst streichen \\
    Betreuer der Ausbildungsfirma \> \> Name wenn relevant, sonst
    streichen \\
    Betreuer der Dualen Hochschule \> \> Prof. Dr.-Ing. Vorname Nachname    
  \end{tabbing}
\end{titlepage}
\vspace*{\fill}




%%% Local Variables:
%%% mode: latex
%%% TeX-master: "main"
%%% End:


\thispagestyle{empty}

\section*{Erklärung}

\vspace*{2cm}

Ich erkläre hiermit ehrenwörtlich: \\
\begin{enumerate}
\item dass ich meine Studienarbeit mit dem Thema
{\itshape \glqq Der Titel\grqq } ohne fremde Hilfe angefertigt habe;
\item dass ich die Übernahme wörtlicher Zitate aus der Literatur sowie
  die Verwendung der Gedanken anderer Autoren an den entsprechenden
  Stellen innerhalb der Arbeit gekennzeichnet habe;
\item dass ich meine Studienarbeit bei keiner anderen Prüfung
  vorgelegt habe;
\item dass die eingereichte elektronische Fassung exakt mit der
  eingereichten schriftlichen Fassung übereinstimmt.
\end{enumerate}
Ich bin mir bewusst, dass eine falsche Erklärung rechtliche Folgen
haben wird.

\vspace*{2cm}

Karlsruhe, \today

\vspace*{2cm}

Mein Name


% Inhaltsverzeichnis
\tableofcontents

\chapter{Einleitung}

Das ist jetzt die Einleitung.

Und das ist die Einleitung mit Umlauten äüöÄÜÖßß

\begin{figure}
  \centering
  \includegraphics[width=3cm]{dhbw_logo}
  \caption{Logo der DHBW als Beispiel für erstes Bild, Bilder
    haben Unterschriften}
  \label{bild1}
\end{figure}

\blindtext[2]


\chapter{Erstes Kapitel}

Das ist jetzt das erste Kapitel. Es werden verschiedenen Textstellen
heraun gezogen \cite{3gpp,Ell08,aks93}. Und manchmal muss man mit
\index{Abkürzungen} arbeiten: \gls{g:kpi} und \gls{DoS}. Ein Bild wird als
Abb.~\ref{bild1} referenziert, eine Tabelle mit Tab.~\ref{tab:tabelle}.

\begin{table}
  \centering
  \caption{Eine Tabelle, Tabellen haben Überschriften!}
  \label{tab:tabelle}
  \begin{tabular}{|c|c|}
    links & rechts \\ \hline
    1 & 2 \\
    3 & 4 \\
  \end{tabular}
\end{table}

\blindtext[4]




% Hier mit Include weitere Kapitel bzw. .tex-files einbinden
% ...

% Anahng mit verschiedenen Listen
\appendix
\addcontentsline{toc}{chapter}{Literaturverzeichnis}
\printbibliography
\printglossaries
\clearpage
\addcontentsline{toc}{chapter}{Index}
\printindex
% ggf. hier Tabelle mit Symbolen 
% (kann auch auf das Inhaltsverzeichnis folgen)
\addcontentsline{toc}{chapter}{Abbildungsverzeichnis}
\listoffigures
\addcontentsline{toc}{chapter}{Tabellenverzeichnis}
\listoftables 
  
\end{document}

%%% Local Variables: 
%%% mode: latex
%%% TeX-master: t
%%% End: 
